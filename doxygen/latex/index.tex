\hypertarget{index_Opis_Ogólny}{}\section{Opis Ogólny}\label{index_Opis_Ogólny}

\begin{DoxyItemize}
\item Najpierw wyliczana jest nowa pozycja na podstawie kroku czasu i prędkości. Prędkość to poprzednia prędkość, grawitacja i lepkość.
\item Potem cząstki dzielone są na komurki według ich położenia.
\item Dla każdej cząstki wyliczana jest siła jaka na nią zadziała. Składa się ona ze zderzeń ze wszystkimi cząstkami w zasięgu, muszą one być w tej samej komurce, albo w sąsiedniej.
\item Prędkość jest wyliczana z poprzedniej prękości i siły. 
\end{DoxyItemize}